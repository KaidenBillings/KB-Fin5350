\documentclass[11pt,]{article}
\usepackage[left=1in,top=1in,right=1in,bottom=1in]{geometry}
\newcommand*{\authorfont}{\fontfamily{phv}\selectfont}
\usepackage[]{mathpazo}


  \usepackage[T1]{fontenc}
  \usepackage[utf8]{inputenc}



\usepackage{abstract}
\renewcommand{\abstractname}{}    % clear the title
\renewcommand{\absnamepos}{empty} % originally center

\renewenvironment{abstract}
 {{%
    \setlength{\leftmargin}{0mm}
    \setlength{\rightmargin}{\leftmargin}%
  }%
  \relax}
 {\endlist}

\makeatletter
\def\@maketitle{%
  \newpage
%  \null
%  \vskip 2em%
%  \begin{center}%
  \let \footnote \thanks
    {\fontsize{18}{20}\selectfont\raggedright  \setlength{\parindent}{0pt} \@title \par}%
}
%\fi
\makeatother




\setcounter{secnumdepth}{0}



\title{What Should Central Bankers Do?  }



\author{\Large Kaiden Billings\vspace{0.05in} \newline\normalsize\emph{Utah State University}  }


\date{}

\usepackage{titlesec}

\titleformat*{\section}{\normalsize\bfseries}
\titleformat*{\subsection}{\normalsize\itshape}
\titleformat*{\subsubsection}{\normalsize\itshape}
\titleformat*{\paragraph}{\normalsize\itshape}
\titleformat*{\subparagraph}{\normalsize\itshape}


\usepackage{natbib}
\bibliographystyle{apsr}



\newtheorem{hypothesis}{Hypothesis}
\usepackage{setspace}

\makeatletter
\@ifpackageloaded{hyperref}{}{%
\ifxetex
  \usepackage[setpagesize=false, % page size defined by xetex
              unicode=false, % unicode breaks when used with xetex
              xetex]{hyperref}
\else
  \usepackage[unicode=true]{hyperref}
\fi
}
\@ifpackageloaded{color}{
    \PassOptionsToPackage{usenames,dvipsnames}{color}
}{%
    \usepackage[usenames,dvipsnames]{color}
}
\makeatother
\hypersetup{breaklinks=true,
            bookmarks=true,
            pdfauthor={Kaiden Billings (Utah State University)},
             pdfkeywords = {central banks, federal reserve, nominal GDP targeting},  
            pdftitle={What Should Central Bankers Do?},
            colorlinks=true,
            citecolor=blue,
            urlcolor=blue,
            linkcolor=magenta,
            pdfborder={0 0 0}}
\urlstyle{same}  % don't use monospace font for urls



\begin{document}
	
% \pagenumbering{arabic}% resets `page` counter to 1 
%
% \maketitle

{% \usefont{T1}{pnc}{m}{n}
\setlength{\parindent}{0pt}
\thispagestyle{plain}
{\fontsize{18}{20}\selectfont\raggedright 
\maketitle  % title \par  

}

{
   \vskip 13.5pt\relax \normalsize\fontsize{11}{12} 
\textbf{\authorfont Kaiden Billings} \hskip 15pt \emph{\small Utah State University}   

}

}







\begin{abstract}

    \hbox{\vrule height .2pt width 39.14pc}

    \vskip 8.5pt % \small 

\noindent Buchanan's approach what economists should do is used as a model to
analyze what central banks should do. An anlysis is given of inflation
targeting and nominal GDP targeting approaches to monetary policy.
Central banks can leverage information markets to influence monetary
policy decisions, but it may be difficult for monetary tools to
effectively influence GDP.


\vskip 8.5pt \noindent \emph{Keywords}: central banks, federal reserve, nominal GDP targeting \par

    \hbox{\vrule height .2pt width 39.14pc}



\end{abstract}


\vskip 6.5pt

\noindent \doublespacing \begin{quote}
\end{quote}

The Federal Reserve may be the most powerful financial institution in
the world. It has supervisory power over many of the world's largest
banks and the ability to directly influence the money supply of the
largest economy in the world. This influence spreads to markets and
economies around the world. Due to this profound influence on the world
economy, there is constant debate on the role of the Federal Reserve and
how monetary policy is conducted. As James Buchanan wrote, ``improvement
must, therefore, be sought in reforms in process, in institutional
change that will allow the operations of politics to mirror more
accurately that set of results that are preferred by those who
participate\ldots{} the constitution of policy rather than policy itself
becomes the relevant object for reform (Buchanan 1987).'' This paper
will reflect this approach to analyzing the processes that lead to
monetary policy decisions and contribute to the question of what central
bankers should do.

\section{The Information Problem}\label{the-information-problem}

The Federal Reserve relies on statistical aggregates of data and
macroeconomic models to inform their policy decisions. There is an
inherent information problem with monetary policy. There is no way 12
people, or 1,200 people, can process the information necessary to
correctly assess the movements of an economy influenced by the millions
of decisions made by people around the world. As Hayek stated, the
knowledge of the circumstances of time and place, held and acted upon by
every individual in the world, make up an integral part of how the
economy keeps moving. ``The continuous flow of goods and services is
maintained by constant deliberate adjustments, by new dispositions made
every day in the light of circumstances not known the day before, by B
stepping in at once when A fails to deliver (Hayek).'' The combination
of aggregated information and the delayed effect of monetary policy on
the economy can lead to undesirable outcomes. By the time macroeconomic
metrics or models signal the need for sudden changes, it may be too late
for monetary policy to make a difference.

There is another information problem on the other side of the equation.
Because the Federal Reserve has so much power to influence the economy,
there is an extreme focus on every communication that comes from the
central bank. Not only are the press releases and minutes from the FOMC
examined at the molecular level, every time a member of the FOMC says
anything about monetary policy, the world is listening, and markets
move. Financial markets are starving for information on what the Fed is
going to do. Members of the FOMC are reluctant to be candid because
anything they say can have a huge impact on financial markets. Despite
efforts to make FOMC meetings and expectations for future interest rates
more transparent, there is still a lot of ambiguity in what influences
rate changes.

\section{Monetary Policy Frameworks}\label{monetary-policy-frameworks}

The current policy framework of inflation targeting stems directly from
the Federal Reserve's mandate to maintain stable prices. The main
reasoning for using inflation targeting is that, ``Most macroeconomists
agree that, in the long run, the inflation rate is the only
macroeconomic variable that monetary policy can affect (Bernanke).''
Although inflation targeting provides a public signal of Federal Reserve
policy, the process is still somewhat of a black box. Inflation
targeting gives the Federal Reserve latitude on how to achieve the
target rate and how to respond to crisis situations. The Federal
Reserve's response to the recession that started in 2008 demonstrated
the range of possible actions. This flexibility, while very useful when
faced with a crisis, is a significant part of why communications on
monetary policy are often murky. The FOMC tries to paint a general
picture of the expected path rates will take through the dot plot and
press releases. However, in the event of sudden changes, it is
impossible for market participants to know how the federal reserve will
react because there is no clear framework for what will influence
policy.

Hayek writes about how a price system can communicate vast amounts of
information through a single number. Nominal GDP targeting is a type of
rules-based policy that ties monetary policy to market expectations for
nominal GDP through a price system. There is a nominal GDP futures
market that acts as an information market for expectations of GDP. The
Federal Reserve offers to buy and sell an unlimited number of contracts
at a price of 1 + the target GDP growth rate. Market participants then
vote with their dollars on what they think the actual GDP will be. The
spot price of the contracts sends a signal to both market participants
and the Federal Reserve about GDP expectations. It communicates to the
Federal Reserve whether they need to expand or contract the money supply
and at the same time communicates to market participants the likely
policy response. This policy approach takes advantage of the knowledge
of the circumstance of time and place that cannot be fed into a
macroeconomic model and allows the ``man on the spot'' to contribute his
knowledge to the information market (Hayek).

A nominal GDP futures market is a very effective way to communicate
information about expectations of economic performance to both market
participants and the Federal Reserve. However, this policy approach
relies on the Federal Reserve's ability to influence GDP through its
monetary tools. One might expect that a growth in the money supply also
causes a growth in aggregate demand. However, changes in the money
supply may not have a causal effect on GDP growth, as is assumed by
proponents of nominal GDP targeting. Research from the St.~Louis Fed did
find a correlation between growth of the money supply and increased GDP
with a lag time of around 3 years. ``At the 12-quarter horizon, for
every 1 percent increase in money base growth, there is about 0.4
percent corresponding increase in GDP growth (Wen).'' Research showing a
strong causal relationship between changes in the money supply and
changes in GDP must be found before the Federal Reserve adopts nominal
GDP targeting as its policy approach.

\section{Conclusion}\label{conclusion}

So, what should central bankers do? Inflation targeting has proven an
effective tool to maintain stable prices. However, the ambiguity of how
policy decisions are made and inefficient communication of information
to markets are significant disadvantages. The market-based nominal GDP
targeting approach provides much clearer information on expectations of
future monetary policy. The crucial question is if the Federal Reserve's
tools to change the money supply can also effectively influence GDP
growth. If so, central bankers should adopt nominal GDP targeting as
their policy approach. It is an effective way to maintain stable growth
in the economy and clearly communicate information between market
participants and central bankers.

\newpage
\singlespacing 
\bibliography{master.bib}

\end{document}
