\documentclass[11pt,]{article}
\usepackage[left=1in,top=1in,right=1in,bottom=1in]{geometry}
\newcommand*{\authorfont}{\fontfamily{phv}\selectfont}
\usepackage[]{mathpazo}


  \usepackage[T1]{fontenc}
  \usepackage[utf8]{inputenc}



\usepackage{abstract}
\renewcommand{\abstractname}{}    % clear the title
\renewcommand{\absnamepos}{empty} % originally center

\renewenvironment{abstract}
 {{%
    \setlength{\leftmargin}{0mm}
    \setlength{\rightmargin}{\leftmargin}%
  }%
  \relax}
 {\endlist}

\makeatletter
\def\@maketitle{%
  \newpage
%  \null
%  \vskip 2em%
%  \begin{center}%
  \let \footnote \thanks
    {\fontsize{18}{20}\selectfont\raggedright  \setlength{\parindent}{0pt} \@title \par}%
}
%\fi
\makeatother




\setcounter{secnumdepth}{0}



\title{Proposal: What Should Central Bankers Do?  }



\author{\Large Kaiden Billings\vspace{0.05in} \newline\normalsize\emph{Utah State University}  }


\date{}

\usepackage{titlesec}

\titleformat*{\section}{\normalsize\bfseries}
\titleformat*{\subsection}{\normalsize\itshape}
\titleformat*{\subsubsection}{\normalsize\itshape}
\titleformat*{\paragraph}{\normalsize\itshape}
\titleformat*{\subparagraph}{\normalsize\itshape}


\usepackage{natbib}
\bibliographystyle{apsr}



\newtheorem{hypothesis}{Hypothesis}
\usepackage{setspace}

\makeatletter
\@ifpackageloaded{hyperref}{}{%
\ifxetex
  \usepackage[setpagesize=false, % page size defined by xetex
              unicode=false, % unicode breaks when used with xetex
              xetex]{hyperref}
\else
  \usepackage[unicode=true]{hyperref}
\fi
}
\@ifpackageloaded{color}{
    \PassOptionsToPackage{usenames,dvipsnames}{color}
}{%
    \usepackage[usenames,dvipsnames]{color}
}
\makeatother
\hypersetup{breaklinks=true,
            bookmarks=true,
            pdfauthor={Kaiden Billings (Utah State University)},
             pdfkeywords = {central banks, federal reserve, machine learning},  
            pdftitle={Proposal: What Should Central Bankers Do?},
            colorlinks=true,
            citecolor=blue,
            urlcolor=blue,
            linkcolor=magenta,
            pdfborder={0 0 0}}
\urlstyle{same}  % don't use monospace font for urls



\begin{document}
	
% \pagenumbering{arabic}% resets `page` counter to 1 
%
% \maketitle

{% \usefont{T1}{pnc}{m}{n}
\setlength{\parindent}{0pt}
\thispagestyle{plain}
{\fontsize{18}{20}\selectfont\raggedright 
\maketitle  % title \par  

}

{
   \vskip 13.5pt\relax \normalsize\fontsize{11}{12} 
\textbf{\authorfont Kaiden Billings} \hskip 15pt \emph{\small Utah State University}   

}

}







\begin{abstract}

    \hbox{\vrule height .2pt width 39.14pc}

    \vskip 8.5pt % \small 

\noindent Buchanan's analysis of what economists should do is used as a model to
analyze what central banks should do. Central banks can use the power of
machine learning to study markets and identify factors that can inform
their policy decisions. An anlysis is given of current policy tools and
how, specifically, the Federal Reserve can fulfill its dual mandate
without basing policy decisions on resource allocation economics.


\vskip 8.5pt \noindent \emph{Keywords}: central banks, federal reserve, machine learning \par

    \hbox{\vrule height .2pt width 39.14pc}



\end{abstract}


\vskip 6.5pt

\noindent \doublespacing \begin{quote}
\end{quote}

\section{Introduction}\label{introduction}

Some may consider the Federal Reserve and other central banks to the
epitome of what Buchanan critizes as he talks about the pitfalls of
resource allocation economics. Central banks have a large influence on
markets around the world. Their existence is central to the monetary
stabililty of many modern countries. Even a centralized institution such
as a central bank can learn from the thoughts of economists such as
Buchanan and Hayek. Central banks can leverage the analytical power of
machine learning to study markets and the exchange between people to
better inform their policy decisions.

\subsection{\texorpdfstring{\emph{What Should Economists Do?}, an
analysis of Buchanan's
address.}{What Should Economists Do?, an analysis of Buchanan's address.}}\label{what-should-economists-do-an-analysis-of-buchanans-address.}

In this address, Buchanan wants to persuade economists to change the
lens through which they view the work they do. He argues that many
economists center their work around resource allocation when they should
focus on simply studying the voluntary exchanges between people. When he
states that he proposes an ``adoption of a sophistaced `catallactics',''
he means that economists should study markets as people who make
voluntary exchanges and not a means of solving the economic problem. The
definition of the economic problem is central to Buchanan's criticism of
the practices of economists. As he writes, ``The economic problem
involves the allocation of scarce means among alternative or competing
ends. The problem is one of allocation, made necessary by the fact of
scarcity, the necessity to choose \citet{buchanan1964}.'' Buchanan
refers to T.D. Wheldon as saying that the very existence of a problem
means that there must be a solution. It is this pursuit of the solution
that has lead economists down an erroneous path. If there were a single
solution to the economic problem, economists may be better defined as
applied mathematicians whose research equates to finding the parameters
that lead to the optimal solution. This work is better left to computers
than humans. Howerever, as Buchanan states: ``If the utility function of
thechoosing agent is fully defined in advance, choice becomes purely
mechanical. No `decision,' as such, is required; there is no weighing of
alternatives. On the other hand, if the utility function is not wholly
defined, choice becomes real, and decisions become unpredictable mental
events \citet{buchanan1964}.'' He goes further to say that economists
should not look through the lens of ``choices'' at all but focus on
``exchange''. He proposes that the word ``Symbiotics'' may be a better
fit to describe the methodology he suggests. The subject of research is
the same but the way economists view it would be completely different.
This method of study focus purely on the obervation of human exchanges.

\subsection{Literature Review}\label{literature-review}

\emph{The Use of Knowledge in Society}

Hayek addresses the way each person their personal knowledge to make
decisions that lead to better economic planning than would be possible
by a central planning organization. Individuals use the unique knowledge
of the circumstance of their time and place to make decisions. This type
of knowledge is not scientific and could not be communicated to a
central entity. The signal given by prices helps individuals to make
economic decisions without knowing the cause or effect of the changes
that occur relevant to their particular decision. These constant,
individual decisions, based on personal knowledge and the signals of
prices lead to a ``conscious direction'' without the individual agents
knowing how they have contributed.

\emph{Machine Learning at Central Banks}

The authors explore the possible applications of machine learning
systems in central banking. As the role of market oversight by central
banks has expanded following the financial crisis of 2008, central banks
have been given access to large amounts of granular data from the
markets they oversee. The central bank's concern for macro-economic
movement can be better understood by aggregating microeconomic
transactions rather than relying on assumptions built into
macro-economic models. In this application, machine learning analysis
can be immensely useful. Machine learning systems can process vast
amounts of data to determine its effects on a certain outcome or search
for patterns in the data. This can be a powerful tool for determining
the ever changing factors that influence macro-economic movements. In
today's evolving economies It is hard to rely on static economic models
to describe or try to predict movements in current economies.

The authors also highlight three challenges to using machine learning
systems in central banks. The first is the ``black box'' problem.
Different machine learning systems have varying degrees of opacity as to
how an output is determined given the different inputs and parameters.
In general, the more accurate systems are more opaque in how inputs are
obtained. Due to the increasing desire for central banks to be
transparent in how policy decisions are made, simpler models would need
to be used in order defend how information is processed and a conclusion
is reached. The second is the problem of correlation vs.~causality.
Machine learning systems are very good at finding correlated patterns in
large data sets. However, it still requires human research and
deliberation to determine a cause and effect relationship. The third
problem is that many machine learning models largely ignore the flow of
time in calculations.

A large portion of the paper goes through the details of how different
machine learning systems work. There are also three case studies done to
illustrate the application of different types of systems to problems
relevent to central banks. The case studies include banking supervision
under imperfect information, UK CPI inflation forecasting, and unicorns
in financial technology.

\subsection{Outline}\label{outline}

\emph{Introduction +Thesis statement: Central banks can leverage the
analytical power of machine learning to study markets and the exchange
between people to better inform their policy decisions. }The dual
mandate of the Federal Reserve +stable prices +Does inflation targeting
adequately maintain stable prices? +Is it a tool of resource allocation
as critized by Buchanan? +maximum employment \emph{Machine learning at
central banks +The use of unsupervised machine learning to find relevant
data pertaining to the dual mandate and inform policy decisions. }The
price of inflation +explore how an adequate price mechanism could be
used to maintain a stable rate of inflation *Conclusion

\newpage
\singlespacing 
\bibliography{./master.bib}

\end{document}
